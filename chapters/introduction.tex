
\chapter{Introduction}
\label{ch:intro}

Page budget for Introduction: 3-5 pages

\section{Background and Motivation}
\label{sec:background}
\begin{itemize}
    \item Awaken the reader's interest and convince her why the theme is important.
    \item Background information might be historical in nature, or it might refer to previous research or practical considerations.
    \item Provide an example or use case for the problem.
    \item It should be written on a level that it's understandable by anyone with a computer science master's degree.
    \item It might contain a small handful of citations if it is needed to justify some main claims or assumptions, but this is not the part to detail any related work and/or compare among each other.
\end{itemize}

\section{Objectives}
\label{sec:objectives}
\begin{itemize}
   \item Define the goals of your study. It might be presented as a bullet list.
  \item Structure your goal by Research Questions (RQs). Describe each of the problems to address in your work and formulate for it a clear research question.``The problem of bla is about$\dots$, We formulate the following RQ1 Can we provide a method for bla such that bla bla''
\end{itemize}


\section{Approach and Contributions}
\label{sec:approach}

\begin{itemize}
  \item Give a brief summary of your overall approach.
  \item Summarize the specific contributions that you made in this thesis (e.g., a task definition, a method or model, a test collection, empirical results, analysis, etc.). It might be presented as a bullet list.
\end{itemize}

\section{Outline}
\begin{itemize}
    \item Give an overview of the main points and the structure of your thesis. ``Chapter 2 covers ... Chapter 3 describes ... ''
    \item Show in a natural way how the different parts (chapters) relate to each other.
\end{itemize}