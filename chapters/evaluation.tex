
\chapter{Experimental Evaluation}
\label{ch:eval}
Page budget for Evaluation: 10-15 pages
\begin{itemize}
    \item Detail your evaluation methodology, present your results, and provide an analysis of them. Results can be quantitative and/or qualitative (from benchmark, user study, user satisfaction survey, etc.).
    \item It is strongly desired that you have some results, nevertheless, this may not be applicable to all types of theses.
\end{itemize}

\section{Experimental Setup}
\begin{itemize}
    \item Explain the methodology used for evaluating your contribution, and the metrics used for evaluation.
    \item If you use any dataset, explain it, detail its version, and mention briefly some main statistics about it, of interest for your problem (e.g., size, provenience, etc.), if appropriate.
    \item If you collect ground truth data, describe your annotation experiment. Explain what the annotators were asked to do (and show a screenshot or schema if available). Detail the number of annotators, their nature (experts, or crowdworkers), the criteria for deciding on each annotation instance (e.g., majority class, dynamic judgments, etc.), the criteria for ensuring quality (e.g., minimum accuracy, filters). If possible, report the inter-annotator agreement coefficient and mention how strong this value means that the agreement is.
\end{itemize}

\section{Experimental Results}
\begin{itemize}
    \item Present the results, using tables and (pretty) plots.
\end{itemize}

\section{Analysis}
Now that you presented the results, what do these results actually mean (esp. regarding the objectives you set out in the introduction)? Can you identify success and failure cases? What do the results say for individual parts you evaluate and overall in combination? Make sure you formulate clear take-home messages.

